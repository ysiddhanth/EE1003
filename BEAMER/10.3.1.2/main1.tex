\documentclass{beamer}
\mode<presentation>
\usepackage{amsmath}
\usepackage{amssymb}
%\usepackage{advdate}
\usepackage{adjustbox}
\usepackage{subcaption}
\usepackage{enumitem}
\usepackage{multicol}
\usepackage{mathtools}
\usepackage{listings}
\usepackage{xcolor}

\definecolor{mygray}{rgb}{0.5,0.5,0.5}
\definecolor{mymauve}{rgb}{0.58,0,0.82}
\definecolor{myblue}{rgb}{0.13,0.13,0.6}

\lstset{
	language=Python,
	backgroundcolor=\color{white},
	commentstyle=\color{mygray},
	keywordstyle=\color{myblue},
	numberstyle=\tiny\color{mygray},
	stringstyle=\color{mymauve},
	basicstyle=\ttfamily\small, % Change font size here
	breaklines=true,
	numbersep=8pt,
	showstringspaces=false,
	tabsize=4
}
\usepackage{url}
\def\UrlBreaks{\do\/\do-}
\usetheme{Boadilla}
\usecolortheme{lily}
\setbeamertemplate{footline}
{
  \leavevmode%
  \hbox{%
  \begin{beamercolorbox}[wd=\paperwidth,ht=2.25ex,dp=1ex,right]{author in head/foot}%
    \insertframenumber{} / \inserttotalframenumber\hspace*{2ex} 
  \end{beamercolorbox}}%
  \vskip0pt%
}
\setbeamertemplate{navigation symbols}{}

\providecommand{\nCr}[2]{\,^{#1}C_{#2}} % nCr
\providecommand{\nPr}[2]{\,^{#1}P_{#2}} % nPr
\providecommand{\mbf}{\mathbf}
\providecommand{\pr}[1]{\ensuremath{\Pr\left(#1\right)}}
\providecommand{\qfunc}[1]{\ensuremath{Q\left(#1\right)}}
\providecommand{\sbrak}[1]{\ensuremath{{}\left[#1\right]}}
\providecommand{\lsbrak}[1]{\ensuremath{{}\left[#1\right.}}
\providecommand{\rsbrak}[1]{\ensuremath{{}\left.#1\right]}}
\providecommand{\brak}[1]{\ensuremath{\left(#1\right)}}
\providecommand{\lbrak}[1]{\ensuremath{\left(#1\right.}}
\providecommand{\rbrak}[1]{\ensuremath{\left.#1\right)}}
\providecommand{\cbrak}[1]{\ensuremath{\left\{#1\right\}}}
\providecommand{\lcbrak}[1]{\ensuremath{\left\{#1\right.}}
\providecommand{\rcbrak}[1]{\ensuremath{\left.#1\right\}}}
\theoremstyle{remark}
\newtheorem{rem}{Remark}
\newcommand{\sgn}{\mathop{\mathrm{sgn}}}
\providecommand{\abs}[1]{\left\vert#1\right\vert}
\providecommand{\res}[1]{\Res\displaylimits_{#1}} 
\providecommand{\norm}[1]{\lVert#1\rVert}
\providecommand{\mtx}[1]{\mathbf{#1}}
\providecommand{\mean}[1]{E\left[ #1 \right]}
\providecommand{\fourier}{\overset{\mathcal{F}}{ \rightleftharpoons}}
%\providecommand{\hilbert}{\overset{\mathcal{H}}{ \rightleftharpoons}}
\providecommand{\system}{\overset{\mathcal{H}}{ \longleftrightarrow}}
	%\newcommand{\solution}[2]{\textbf{Solution:}{#1}}
%\newcommand{\solution}{\noindent \textbf{Solution: }}
\providecommand{\dec}[2]{\ensuremath{\overset{#1}{\underset{#2}{\gtrless}}}}
\newcommand{\myvec}[1]{\ensuremath{\begin{pmatrix}#1\end{pmatrix}}}
\newcommand{\mydec}[1]{\ensuremath{\begin{vmatrix}#1\end{vmatrix}}}
\let\vec\mathbf

\lstset{
%language=C,
frame=single, 
breaklines=true,
columns=fullflexible
}

\numberwithin{equation}{section}

\title{NCERT: 10.3.1.2}
\author{Y Siddhanth \\ EE24BTECH11059\\ EE1030}

\date{\today} 
\begin{document}

\begin{frame}
\titlepage
\end{frame}

\section*{Outline}
\begin{frame}
\tableofcontents
\end{frame}
\section{Problem}
\begin{frame}
\frametitle{Problem Statement}
The coach of a cricket team buys 3 bats and 6 balls for Rs.3900. Later, she buys another bat and 3 more balls of the same kind for Rs.1300. Find the price of the ball and the bat using LU factorization. \\
\end{frame}

\section{Solution}
\subsection{Converting to Matrix Form}
\begin{frame}
	\frametitle{Converting to Matrix Form}
	First, we rewrite the question as a system of linear equations.
	\begin{align}
		x_1 &\implies \text{bat} \\
		x_2 &\implies \text{ball}
	\end{align}
	\begin{align}
		3x_1 + 6x_2 &= 3900 \\
		x_1 + 3x_2 &= 1300 
	\end{align}
	Now, converting into a matrix form, we get:
	\begin{align}
		\myvec{3&6\\1&3}\myvec{x_1 \\ x_2} &= \myvec{3900 \\ 1300} \\ 
		\vec{A}x &= \vec{b}
	\end{align}
\end{frame}
\subsection{LU-Decomposition}
\begin{frame}
	\frametitle{LU-Decomposition}
	To solve the above equation, we have to apply LU - Factorization to matrix $\vec{A}$. \\
	We do so because, 
	\begin{align}
		A &\rightarrow LU \\
		L &\rightarrow \text{Lower Triangular Matrix} \\
		U &\rightarrow \text{Upper Triangular Matrix}
	\end{align}
	Let $y = \vec{U}x$, then we can rewrite the above equation as,
	\begin{align}
		\vec{A}x = \vec{b} \implies \vec{LU}x = \vec{b} \implies \vec{L}y = \vec{b}
	\end{align}
	Now, the above equation can be solved using front-substitution since $\vec{L}$ is lower triangular, thus we get the solution vector $y$. \\
\end{frame}
\subsection{LU-Decomposition}
\begin{frame}
	\frametitle{LU-Decomposition}
	Using this we solve for $x$ in $y = \vec{U}x$ using back-substitution knowing that $\vec{U}$ is upper triangular. LU Factorizing $\vec{A}$, we get:
	\begin{align}
		\vec{A} &= \myvec{1&0\\0.33&1}\myvec{3&6\\0&1} \\ 
		\vec{L} &= \myvec{1&0\\0.33&1} \\
		\vec{U} &= \myvec{3&6\\0&1}
	\end{align}
	The solution can now be obtained as:
	\begin{align}
		\myvec{1&0\\0.33&1}\myvec{y_1 \\ y_2} &= \myvec{3900 \\ 1300}
	\end{align}
	Solving for y, we clearly get
	\begin{align}
		\myvec{y_1 \\ y_2} = \myvec{3900 \\ 0}
	\end{align}
\end{frame}
\subsection{LU-Decomposition}
\begin{frame}
	\frametitle{LU-Decomposition}
	Now, solving for x via back substitution, 
	\begin{align}
		\myvec{3&6\\0&1}\myvec{x_1 \\ x_2} &= \myvec{3900 \\ 0} \\
		\myvec{x_1 \\ x_2} &= \myvec{1300 \\ 0} 
	\end{align}
	Thus, the price of the bat ($x_1$) and the ball ($x_2$) are obtained.\\
\end{frame}
\subsection{Doolittle's Algorithm}
\begin{frame}
	\frametitle{Doolittle's Algrorithm}
	The LU decomposition can be efficiently computed using Doolittle's algorithm. This method generates the matrices \( L \) (lower triangular) and \( U \) (upper triangular) such that \( A = LU \). The elements of these matrices are calculated as follows: \\
	Elements of the \( U \) Matrix:  \\
	For each column \( j \):
	\begin{align}
		U_{ij} &= A_{ij} \quad \text{if } i = 1, \\
		U_{ij} &= A_{ij} - \sum_{k=0}^{i-1} L_{ik} U_{kj} \quad \text{if } i > 1.
	\end{align}
	Elements of the \( L \) Matrix: \\
	For each row \( i \):
	\begin{align}
		L_{ij} &= \frac{A_{ij}}{U_{jj}} \quad \text{if } j = 1, \\
		L_{ij} &= \frac{A_{ij} - \sum_{k=0}^{j-1} L_{ik} U_{kj}}{U_{jj}} \quad \text{if } j > 1.
	\end{align}
\end{frame}
\begin{frame}
	\begin{align}
		\begin{bmatrix}
			a_{11} & a_{12} &a_{13} \\
			a_{21} & a_{22} & a_{23} \\
			a_{31} & a_{32} & a_{33}
		\end{bmatrix}
		&=
		\begin{bmatrix}
			1 & 0 & 0 \\
			l_{21} & 1 & 0 \\
			l_{31} & l_{32} & 1
		\end{bmatrix}
		\begin{bmatrix}
			u_{11} & u_{12} & u_{13} \\
			0 & u_{22} & u_{23} \\
			0 & 0 & u_{33}
		\end{bmatrix} \\ 
		\begin{bmatrix}
		a_{11} & a_{12} &a_{13} \\
		a_{21} & a_{22} & a_{23} \\
		a_{31} & a_{32} & a_{33}
		\end{bmatrix}
		&=
		\begin{bmatrix}
			u_{11} & u_{12} & u_{13} \\
			l_{21}u_{11} & l_{21}u_{12} + u_{22} & l_{21}u_{13} + u_{23} \\
			l_{31}u_{11} & l_{31}u_{12} + l_{32}u_{22} & l_{31}u_{13} + l_{32}u_{23} + u_{33}
		\end{bmatrix}
	\end{align}
\end{frame}
\end{document}
